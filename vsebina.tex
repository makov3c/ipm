
\subsection*{Algebrske struktre}
\begin{itemize}
	\item \textbf{grupoid} $(M, \cdot)$ urejen par z neprazno množico $M$ in zaprto opreacijo $\cdot$.
	\item \textbf{polgrupa} grupoid z asociativno operacijo $ \forall x,y,z \in M : (x\cdot y)\cdot z = x\cdot (y\cdot z)$.
	\item \textbf{monoid} polgrupa z enoto $ \exists e \in M \ \forall x \in M : e\cdot x = x\cdot e = x$.
	\item \textbf{grupa} polgrupa v kateri ima vsak element inverz $ \forall x \in M \ \exists x^{-1} \in M : x\cdot x^{-1} = x^{-1}\cdot x = e$.
	\item \textbf{abelova grupa} grupa s komutativno operacijo $ \forall x,y \in M  : x\cdot y = y\cdot x$.
\end{itemize} 

\subsection*{Kolobarji}
\textbf{Kolobar} je množica $R$ skupaj z dvema operacijama (oznaka: $+, \cdot$) tako, da velja:
\begin{itemize}
	\item $(R, +)$ je abelova grupa
	\item $\forall a, b, c \in R\ :\ a(b+c) = ab + ac$ (distrubutivnost)
	\item $\forall a, b, c \in R\ :\ (a+b)c = ac + bc$ (distrubutivnost)
	\item $\forall a, b \in R\ :\ ab \in R$ (zaprtost množenja)
	\item $\forall a, b, c \in R\ :\ (ab)c = a(bc)$ (asociativnost*)
	\item $\exists e \in R\ \forall a \in R\ :\ e\cdot a = a = e\cdot a$ (enota*)
\end{itemize}
Kolobar je \textbf{komutativen}, če $\forall a, b \in R\ :\ ab = ba$.
Kolobar je \textbf{kolobar z deljenjem}, če $\forall a \in R-\{0\}\ \exists a^{-1} \in R :\ aa^{-1} = 1$ element $1$ je \emph{enota kolobarja}.

Kolobar, ki ima vse naštete lastnosti je \textbf{obseg}.

\subsubsection*{Delitelji niča in celi kolobarji}
Naj bo $R$ komutativen koloboar. Tedaj je $a \in R,\ a \neq 0$ \textbf{delitelj niča}, če
\[ \exists b \in R,\ b \neq 0\ :\ ab = 0 \]

\textbf{Cel kolobar} je komutativen kolobar z enoto ($1\neq0$), ki nima deliteljev niča.

\subsection*{Razširitve kolobarjev}
Naj bo $K$ kolobar \textbf{brez enote}:
\begin{gather*}
	\mathbb{Z} \times K = \{ n \in \mathbb{Z}, a \in K \\
	(n,a) + (m,b) = (n+m, a+b) \\
	(n,a) \cdot (m,b) = (nm, nb+am+ab)
\end{gather*}

Naj bo $K$ komutativen kolobar \emph{brez deliteljev niča} vendar niso vsi elementi obrnljivi. Dodamo ulomke definirane kot ekvivalenčne razrede dvojic z ekvivalenčno (\emph{refleksivno, simetrično, tranzitivno}) relacijo $\sim$.
\begin{gather*}
	K \times K-\{0\} \Big/_\sim\\
	\frac{a}{b} \sim \frac{ka}{kb} \quad \forall k \in K-\{0\} \\
	\frac{a}{b} + \frac{a'}{b'} = \frac{ab' + a'b}{bb'}\\
	\frac{a}{b} \cdot \frac{a'}{b'} = \frac{aa'}{bb'}
\end{gather*}
\emph{Če bi bila $b$ in $b'$ delitelja niča, bi imeli težave.}

Tako dobimo \textbf{obseg ulomkov za $K$}.

\subsection*{Wedderburnov izrek}
Končen kolobar brez deliteljev niča je \textbf{obseg}.

Posledica:
$\mathbb{Z}_n$ je obseg $\iff$ $n\in\mathbb{P}$

\subsection*{Karakteristika kolobarja}
\textbf{Karakteristika} kolobarja $R$ je najmanjši $n\in \mathbb{N}$, tako da velja 
\[\forall a \in R\ :\ na = \underbrace{a+a+...+a}_{n\textmd{-krat}} = 0\]
Če tak $n$ ne obstaja je karakteristika enaka $0$.

Če je $1 \in R$, je $\text{char}(R) = \text{red enote}$ oziroma najmanjši $n \in \mathbb{N}$, da je $1\cdot n = 0$. 

Če je $R$ cel kolobar, je $\textmd{char}R \in \{0\} \cup \mathbb{P}$.

\subsection*{Homomorfizem}
Naj bosta $K$, $L$ kolobarja. $f:K \to L$ je \textbf{homomorfizem}, če $\forall a, b \in K$ velja:
\begin{align*}
	f(a+b) &= f(a) + f(b)\\
	f(a \cdot b) &= f(a) \cdot f(b)
\end{align*}

Iz aditivnosti sledi: $f(0) = 0$ in $f(-a) = -f(a)$.


\textbf{Izomorfizem} je bijektivni homomorfizem.

\textbf{Avtomorfizem} je homomorfizem $f: K \to K$.

Če je $f(1) = 1$, pravimo, da je homomorfizem \textbf{unitalen}. Če je unitelen in če je $a$ obrnljiv, potem je $f(a^{-1}) = f(a)^{-1}$.

\subsubsection*{Slika / zaloga vrednosti}
Zaloga vrednosti $f$ je $f(K) = \{ f(a)\ |\ a \in K\} = \ \text{Im}K \leq L$.
\[ f \text{ je surjektiven} \iff \text{Im}f = L\]

\subsubsection*{Jedro / ničelna množica}
Praslika 0 je $f^{-1}(0) = \{ a \in K \ |\ f(a) = 0\} = \text{Ker}f \leq K$.
\[ \forall a \in K, \forall x \in \text{Ker} f:\ f(ax) = f(a)f(x) = 0 \implies \text{Ker}f \lhd K\]

\subsection*{Ideali}
Podkolobar $I \leq K$ je ideal, če velja $I\cdot K \subseteq I$ in $K\cdot I \subseteq I$. Oznaka: $I \lhd K$.

V nekumutativnih kolobarjih ločimo \textbf{leve} in \textbf{desne} kolobarje.

$K$ in $\{0\}$ sta \textbf{neprava ideala}.

V osegih obstajajo le nepravi ideali. Še več, pravi ideali ne vsebujejo obrnljivih elementov.

\subsubsection*{Glavni ideali}
Naj bo $K$ kolobar in $x \in K$.
\[ (x) = Kx = \{kx\ |\ k \in K\}\]

Kolobar je \textbf{glavno idealski}, če se vsi njegovi ideali glavni.

\subsubsection*{Kvocientni ideal}
Za dvostranski ideal $I \lhd K$ definiramo ekvivalenčno relacijo $\sim$:
\[ \forall a,b \in K:\ a \sim b \iff a-b \in I \]
$K$ razdelimo na ekvivalenčne razrede $K/_\sim$, ki pa jih lahko označimo tudi z $K/I$. Ekvivalenčni razred, ki pripada $x \in K$ označimo $[x]$ ali pa $(x+I)$.

Dodamo opreaciji:
\begin{align*}
	(x+I) + (y + I) = (x+y+I) \\
	(x+I) \cdot (y + I) = (x\cdot y+I) \\
\end{align*}
$(K/I, +, \cdot)$ je kolobar in podeduje lastnosti $K$.

\subsection*{Izrek o izomorfizmu}
Naj bo $f: K \to L$ homomorfizem kolobarjev. Potem je $\text{Ker}f \lhd K$ in imamo naravni izomorfizem:
\begin{align*}
	\bar{f}: K/\text{Ker}f \to \text{Im}f\\
	\bar{f} (x + \text{Ker}f) = f(x) \\
	K/\text{Ker}f \cong \text{Im}f
\end{align*}

\section*{Kolobarji polinomov}
\subsection*{Računanje s kompleksnimi števili}
\begin{gather*}
	z = x + iy = r e^{i\varphi} = r\left( \cos \varphi + i \sin \varphi \right)
\end{gather*}
\begin{align*}
	r = |z| = \sqrt{x^2 + y^2} && \varphi = \arg z = \arctan \frac{y}{x}
\end{align*}

\subsubsection*{De Moivreova formula}
\begin{align*}
	z^n = r^n\left( \cos \varphi n + i \sin \varphi n \right)
\end{align*}

\subsubsection*{Osnovni izrek algebre}
Vsak nekonstanten polinom $a_n x^n + \dots + a_0$ ima natanko $n$ kompleksnih ničel (štetih z večkratnostjo).

\subsection*{Trigonometrične identitete}
\begin{align*}
	&\sin(x \pm y) = \sin(x) \cos(y) \pm \cos(x) \sin(y) \\
	&\cos(x \pm y) = \cos(x) \cos(y) \mp \sin(x) \sin(y)\\
	&\tan(x \pm y) = \frac{\tan(x)\pm \tan(y)}{1 \mp \tan(x) \tan(y)}\\
	&\cot(x \pm y) = \frac{\cot(x)\cot(y) \mp 1}{\tan(x) \pm \tan(y)}\\
	&\sin^2(x)+\cos^2(x) = 1\\
	&1+\cot^2(x) = \frac{1}{\sin^2(x)}\\
	&1+\tan^2(x) = \frac{1}{\cos^2(x)}\\
	&\sin\frac{x}{2} = \pm\sqrt{\frac{1-\cos x}{2}}\\
	&\cos\frac{x}{2} = \pm\sqrt{\frac{1+\cos x}{2}}\\
\end{align*}

\subsection*{Mali Fermantov izrek}
\begin{align*}
	\forall a \in \mathbb{Z}, p \in \mathbb{P}:\ a^p \equiv_p a
\end{align*}

\subsection*{Polinomi}
Polinom je \textbf{razcepen}, če ga lahko zapišemo kot produkt dveh nekonstantnih polinomov.
Nekonstanten polinom, ki ni razcepen je \textbf{nerazcepen}.

Polinom $a_n x^n + \dots + a_0$ je \textbf{primitiven}, če velja $\gcd(a_0, \dots, a_n) = 0$

\subsubsection*{Gaussova lema}
\begin{align*}
	p(x) \in \mathbb{Z}[x] \text{ razcepen nad } \mathbb{Z} \iff p(x) \text{ razcepen nad } \mathbb{Q}
\end{align*}

\subsubsection*{Hornerjev algoritem}
\[a_n x^n + ... + a_0 = 0\]
\begin{itemize}
	\item možne cele ničle: $\pm$delitelji $a_0$
	\item možne racionalne ničle: $\pm \frac{\text{delitelji }a_0}{\text{delitelji }a_n} = k$
\end{itemize}
\begin{center}
	\begin{tabular}{ l|l l l l}
			& $a_n$ & $a_{n-1}$ & $...$ & $a_0$ \\ \hline
		$k$ &       & $ka_n$    & $...$ & \\ \hline
			& $a_n$ & $ka_n - a_{n-1}$ & $...$ & ostanek\\
	\end{tabular}
\end{center}

\subsubsection*{Eisensteinov kriterij}
Naj bo $a(x) = a_n x^n + \dots + a_0 \in \mathbb{Z}[x]$ polinom. Če $\exists p \in \mathbb{P} : p | a_0, \dots, a_{n-1} \wedge p \nmid a_n \wedge p^2 \nmid a_0$, potem je $a(x)$ nerazcepen nad $\mathbb{Q}$.

\subsubsection*{Rodovne funkcije}
\[
	\begin{aligned}
		\sum_{n=0}^{\infty} q^n &= \frac{1}{1-q} &
		\sum_{n=0}^{b} q^n &= \frac{1-q^{b+1}}{1-q}
		\\
		\sum_{n=a}^{\infty} q^n &= \frac{q^{a}}{1-q} &
		\sum_{n=a}^{b} q^n &= \frac{q^a-q^{b+1}}{1-q}
	\end{aligned}
\]
\[
	a^n - b^n = (a-b)(a^{n-1} + a^{n-2}b + ... + ab^{n-2} + b^{n-1})  
\]
\[ \textstyle \frac{a_0 + ... + a_{k-1}x^{k-1}}{1-x^k} = a_0 + ... + a_{k-1}x^{k-1} + a_0^k + ... + a_{k-1}x^{2k-1} + ...\]
\[ (x+y)^n = \sum_{k=0}^{n} \binom{n}{k} x^{n-k}y^{k} \]
\[ \frac{1}{(1-x)^n} = \sum_{k=0}^{n} \binom{n+k-1}{k} x^{k} \]
\[ B_\lambda(x) = \sum_{n} \binom{\lambda}{n} x^{n} = (1+x)^\lambda; \qquad \binom{\lambda}{n} = \frac{\lambda^{\underline{n}}}{n!}\]

\subsubsection*{Mobiusova formula}
\begin{align*}
	\mu(n) = \begin{cases}
		1 &n = 1, \\
		0 &\exists p \in P: p^2 | n \\
		(-1)^k &\text{$n$ je produkt $k$ različnih praštevil.}
	\end{cases}
\end{align*}

Število nerazcepnih polinomov v $\mathbb{Z}_p[x]$ stopnje $n$ je enako
\[ N_p(n) = \frac{p-1}{n} \sum_{d|n} \mu(\frac{n}{d}) p^d\]

\subsection*{Eulerjeva funkcija}
\[ 
	\begin{aligned}
		\varphi (n) &= |\{k\in [n] : D(n,k) = 1 \}| \\
				&= \textmd{št. proti $n$ tujih števil, ki so $\leq$ $n$} \\
		\varphi(p) &= p-1 \qquad \qquad p \in \mathbb{P}\\
		\varphi(p^k) &= p^k-p^{k-1} = p^k(1-\frac{1}{p})\\
	\end{aligned}
\]
\[\sum_{d|n} \varphi(d) = n\]

\subsubsection*{Največji skupni delitelj}
Za polinoma $a,b \in F[x]$ obstaja enolično določen največji skupni delitelj $d = \gcd(a,b)$.
\subsubsection*{Razširjen evklidov algoritem}
\begin{algorithm}
vhod: $(a, b)$
($r_0$, $x_0$, $y_0$) = ($a$, 1, 0)
($r_1$, $x_1$, $y_1$) = ($b$, 0, 1)
$i$ = 1

dokler $r_i$ $\neq$ 0:
$i$ = $i$+1
$k_i$ = $r_{i-2} // r_{i-1}$
$(r_i, x_i, y_i)$ = $(r_{i-2}, x_{i-2}, y_{i-2}) - k_i(r_{i-1}, x_{i-1}, y_{i-1})$
konec zanke
vrni: $(r_{i-1}, x_{i-1}, y_{i-1})$
\end{algorithm}

Trojica $(d, x, y)$, ki jo vrne razširjen evklidov algoritem z vhodnim podatkomk $(a, b)$, zadošča:
\[ax + by = d \text{ in } d = \textrm{gcd}(a, b)\] 
