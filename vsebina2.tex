
\section*{Obsegi}
Obseg je \emph{komutativen} kkolobar v katerem so vsi neničelni elementi \emph{obrnljivi}.

\subsection*{Razširitve obsegov}
Če je $K$ podobseg obsega $F$, pravimo, da je $F$ \textbf{razširitev} obsega $K$ in pišemo $K \leq F$.

$F$ je avtomatično tudi vektorski prostor nad $K$ dimenzije $\dim_K(F) = [F:K]$.

Če je $[F:K]$ končna, je $F$ \textbf{končna razširitev}, sicer pa je \textbf{neskončna razširitev}.

\[ K \leq F \leq E \implies [E:K] = [E:F] \cdot [F:K] \]

\begin{itemize}
	\item Najmanjši podkolobar kolobarja $F$, ki vsebuje $K \leq F$ in $a \in F$ je
	\[K[a] = \{p(a)\ |\ p(x) \in K[x]\}\]
	\item Najmanjši podobseg obsega $F$, ki vsebuje $K \leq F$ in $a \in F$ je
	\[K(a) = \left\{ \frac{ p(a) }{ q(a) }\ \middle|\ p(x), q(x) \in K[x] \right\} \]
\end{itemize}

\subsubsection*{Enostavne razširitve obsegov}
\emph{Razširitev je enostavna, če je generirana z enim samim elementom.}

Naj bo $K \leq F$ in $a \in F$. Oglejmo si homomorfizem
\begin{align*}
	f_a: K[x] &\to F \\
		p(x) & \mapsto p(a)
\end{align*}
\begin{align*}
	\text{Im}f_a &= K[a] \\
	\text{Ker}f_a &= \{ p(x) \in K[x]\ |\ p(a) = 0\}
\end{align*}

\renewcommand{\labelitemii}{$\iff$}
\begin{itemize}
	\item $a$ je \textbf{transcendenten} nad $K$
	\begin{itemize}
		\item $a$ ni ničala nobenega neničelnega polinoma iz $K[x]$
		\item $\text{Ker} f_a = (0)$
		\item $f_a$ injektivna
	\end{itemize}
	\item $a$ je \textbf{algebraičen} nad $K$
	\begin{itemize}
		\item $\exists p(x) \in K[x], p \neq 0:\ p(a) = 0$
	\end{itemize}
\end{itemize}

Če so vsi elementi $F$ algebraični nad $K$, je $F$ \textbf{algebraična razširitev}. V nasprotnem primeru pa je $F$ \textbf{transcendentna razširitev}.

Če je $a \in F$ \emph{transcendenten} nad $K$, je
\begin{align*}
	K[a] \cong  K[x] && K(a) \cong K(x)
\end{align*}

Če je $a \in F$ \emph{algebraičen} nad $K$, velja:
\begin{itemize}
	\item $\exists$ natanko določen \textbf{minimalni polinom} $g_a \in K[x]$, ki deli vse polinome z ničlo v $a$.  $g_a$ \textbf{moničen} (\emph{vodilni koef. =1})
	\item $\text{Ker}f_a = (g_a)$
	\item $K(a) = K[a] \cong K[x] / (g_a)$
	\item $[K(a):K] = \deg g_a$, \textbf{stopnja} $a$ nad $K$ (oznaka: $\deg_K a$)
	\item Ideal $(g_a) \lhd K[x]$ je maksimalen $\implies$ $K[x] / (g_a)$ je obseg
\end{itemize}

Naj bo $F$ končna razširitev $K$, potem za vsak $a \in F$ velja 
\[\deg_K (a) \big| [F:K] \]

Vse transcendentne razširitve so neskončne, algebraične pa so lahko končne ali pa neskončne (če dodamo več elementov).

Naj bo $K \leq F$ in $A \subseteq F$ množica števil, ki so algebraična nad $K$. Potem je $K(A)$ algebraična nad $K$.

Naj bo $K \leq F \leq E$, $F$ algebraična nad $K$, $E$ algebraična nad $F$. Potem je $E$ algebraična nad $K$.

\subsection*{Razpadni obseg polinoma}
Razpadni obseg polinoma $p(x)$ nad obsegom $K$ označimo z $K(p(x))$. To je najmanjši podobseg $K$ v katerem je $p(x)$ povsem razcepen (\emph{$K$ vsebuje vse ničle $p(x)$}).

Za vsak $n$ obstaja razširitev stopnje $n$ obsega $\mathbb{Z}_p$. Vsaka taka razširitev je izomorfna $\mathbb{Z}_p (x^{p^n} - x)$.

Edini (do izomorfizma) obseg moči $n^p$ je \textbf{Galoisov obseg} $\text{GF}(p^n)$.

Naj bo $K$ končen kolobar (ne nujno komutativen). Če $K$ nima deliteljev niča, je $|K| = p^n$ in $K \cong \text{GL}(n^p)$.

\subsubsection*{Galoisovi obsegi}
\[\text{GF}(p) \cong \mathbb{Z}_p \qquad p \in \mathbb{P}\]
\[ \text{GF}(p^n) \cong \mathbb{Z}_p[x]/(u) \]
\begin{itemize}
	\item $u \in \mathbb{Z}_p[x]$ je nerazcepen polinom stopnje $n$
	\item elementi $\text{GF}(p^n)$ so ostanki polinomov iz $\mathbb{Z}_p$ pri deljenju z polinomom $u$
	\item seštevanje je enako kot seštevanje v $\mathbb{Z}_p[x]$
	\item produkt izračunamo v $\mathbb{Z}_p[x]$ nato pa vzamemo ostanek pri deljenju z $u$
\end{itemize}

Množica neničelnih/obrnljivih elementov $(GF(p^n)^*, \cdot) \cong (\mathbb{Z}_{p^n-1}, \cdot)$ je vedno izomorfna neki ciklični grupi.
Generatorjem te grupe rečemo \textbf{primitivni elementi} Galoisovega obsega.

\subsection*{Ciklotomski obseg}
je oblike $\mathbb{Q}(e^{\frac{2 \pi i}{n}})$ kjer je $n \in \mathbb{N}$.

\[ [\mathbb{Q}(e^{\frac{2 \pi i}{n}}) : \mathbb{Q}] = \varphi(n) \]

$\varphi$ je Eulerjeva funkcija.

\subsection*{Konstruktibilna števila}
Število $a \in \mathbb{R}$ je konstruktibilno $\iff$
\[a \in F_n \qquad \mathbb{Q} = F_0 \leq \dots \leq F_n\]
kjer je $[F_j : F_{j-1}] = 2$ za $\forall j = 1, \dots, n$.

\emph{Število je konstruktibilno, če leži v zaporedju razširitev stopnje 2.}

\subsection*{Kvaternioni}
\[ \mathbb{H} = \{t+xi+yj+zk\ |\ t,x,y,z \in \mathbb{R}\}\]
Kvaternioni so nekomutativen kolobar z deljenjem.
\begin{center}
	\renewcommand{\arraystretch}{1.5}
	\begin{tabular}{c | c c c c}
		$\cdot$ & $1$ & $i $ & $j $ & $k $ \\ \hline
		$1$ & $1$ & $i $ & $j $ & $k $ \\
		$i$ & $i$ & $-1$ & $k $ & $-j$ \\
		$j$ & $j$ & $-k$ & $-1$ & $i $ \\
		$k$ & $k$ & $j $ & $-i$ & $-1$ \\
	\end{tabular}
\end{center}
\emph{Prvi operand je na začetku vrstice, drugi pa na vrhu stolpca.}

\subsubsection*{Vektorska oblika}
\[ q = t+xi+yj+zk = (t, \vect{r}) \qquad \vect{r} = (x,y,z)\]

Vektorje $\vect{x} = (x_1, x_2, x_3) \in \mathbb{R}^3$ identificiramo s kvaternioni $(0, \vect{x})$, ki imajo skalarni del enak 0. 

Množenje izrazimo s formulo:
\[ q_1 q_2 = (t_1 t_2 - \vect{r_1}\cdot \vect{r_2},\ t_1 \vect{r_2} + t_2 \vect{r_1} + \vect{r_1} \times \vect{r_2})\]
\begin{align*}
	\vect{a} \times \vect{b} = \begin{vmatrix}
		\mathbf{i} & \mathbf{j} & \mathbf{k} \\
		a_1 & a_2 & a_3 \\
		b_1 & b_2 & b_3 \\
	\end{vmatrix}
\end{align*}

Konjugirani kvaternion:
\[ q^* = (t, -\vect{r})\]

Norma kvaterniona:
\[ |q|^2 = q q^* = t^2+x^2+y^2+z^2 = t^2+\|\vect{r}\|^2\]

Inverz kvaterniona:
\[ q^{-1} = \frac{q^*}{|q|^2}\]

\subsubsection*{Vrtenje vektorjev}
Vektor $\vect{x} \in \mathbb{R}^3$ bomo zavrteli okoli osi $\vect{e} \in \mathbb{R}^3$, $|\vect{e}| = 1$ za kot $\varphi \in \mathbb{R}$.

Enotski kvaternioni tvorijo grupo:
\[s^3 = \{(t, \vect{r}) \in \mathbb{H}\ |\ t^2 + \| \vect{r}\|^2 = 1\}\]

Definirajmo enotski kvaternion:
\[ q = \cos\frac{\varphi}{2} + \sin\frac{\varphi}{2} \vect{e} \]

Zavrten vektor je potem:
\[ R(\vect{e}, \varphi) \vect{x} = q \vect{x} q^*\]


Rotacijske matrike so ortogonalne matrike z determinanto 1 in tvorijo grupo:
\[SO(3) = \{ R \in \mathbb{R}^{3\times 3}\ |\ R^TR = I, \det(R) = 1\}\]

Iz rotacijske matrike $R$ lahko izračunamo os rotacije:

Os vrtenja je vzporedna lastnemu vektorju $\vect{e}$ matrike $R$, ki ustreza lastni vrednosti $\lambda = 1$.
Za $\varphi \notin \{0, \pi\}$:
\[\vect{e} = \frac{1}{2 \sin \varphi} \begin{bmatrix}
	R_{32}-R_{23} \\
	R_{13}-R_{31} \\
	R_{21}-R_{12} \\
\end{bmatrix}\]

Kot rotacije pa dobimo s formulo $\cos \varphi = \frac{\text{sl}(R) - 1}{2}$

\section*{Topologija}

Naj bo $X$ poljubna množica. Topologija na $X$ je podana z družino odprtih množic $\tau$, ki je zaprta za \textbf{poljubne unije} in \textbf{končne preseke}.

Prazna unija je prazna množica, prazen presek pa cela množica.

Najmanjša možna topologija je $\tau = \{\emptyset, X\}$ \textbf{trivialna}.

Največja možna topologija je $\tau = P(X)$ \textbf{diskretna}.

\subsubsection*{Topologija glede na metriko}
$d: X \times X \to [0,\infty)$ je metrika, če velja:
\begin{itemize}
	\item $d(x, y) = 0 \iff x = y$
	\item $d(x, y) = d(y, x)$
	\item $d(x, y) + d(y, z) \geq d(x, z)$
\end{itemize}
Topologija iz metrike na $X$ je:
\[ \tau_d = \{ U \subseteq X\ |\ U \text{ odprta glede na } d\}\]

$A$ je \textbf{odprta množica}, če so vse točke notranje ($\forall a \in A\ \exists \varepsilon > 0:\ K(a, \varepsilon) \subseteq A$).

$A$ je \textbf{zaprta množica} $\iff$ $A^{\complement}$ odprta $\iff$ vsebuje vse svoje robne točke. 

Naj bo $A \subseteq X$.
\begin{itemize}
	\item \textbf{Notranjost} $\text{Int}(A) = \mathring{A}$ = največja odprta množica vsebovana v $A$.
	\item \textbf{Zaprtje} $\text{Cl}(A) = \bar{A}$ = najmanjša zaprta množica, ki še vsebuje v $A$ \\
			= presek vseh zaprtih množic, ki vsebujejo $A$
	\item  \textbf{Rob} $\text{Fr}(A) = \partial A = \dot{A}$ = $\text{Cl}(A) - \text{Int}(A)$
\end{itemize}

\subsubsection*{Metrizabilnost}
$(X, \tau)$ je metrizabilen, če obstaja metrika $d$ na $X$, da $\tau = \tau_d$

\subsection*{Zveznost}
Funkcija $f: (X, \tau_X) \to (Y, \tau_Y)$ je zvezna v točki $x \in X$, če lahko za vsako odprto okolico $V$ točke $f(x)$ najdemo odprto okolico $U$ točke $x$, da velja $f(U) \subset V$.

Funkcija $f: (X, \tau_X) \to (Y, \tau_Y)$ je zvezna, če
\begin{multline*}
	\forall x \in X\ \forall \varepsilon > 0\ \forall \delta > 0\ \forall x' \in X:\\ d(x,x') < \delta \implies d(f(x),f(x')) < \varepsilon
\end{multline*}
Ekvivalentna topološka definicija:
\[\forall V \in \tau_Y : f^{-1}(V) \in \tau_X\]
\emph{Funkcija je zvezva, če je praslika vsake odprte množice odprta.}

Naslednje trditve so ekvivalentne:
\begin{itemize}
	\item $f: X \to Y$ je zvezna
	\item $\forall A^{\text{odp}} \subseteq Y:\ f^{-1}(A) \text{ odprta v }X$
	\item $\forall B^{\text{zap}} \subseteq Y:\ f^{-1}(B) \text{ zaprta v }X$
	\item $\forall A \subseteq X:\ f(\bar{A}) \subseteq \overline{f(A)}$
\end{itemize}

\section*{Homeomorfizmi}
$f: (X, \tau_X) \to (Y, \tau_Y)$ je \textbf{homeomorfizem}, če je $f$ bijekcija in sta $f$ in $f^{-1}$ zvezni.

Prostora $(X, \tau_X)$ in $(Y, \tau_Y)$ sta \textbf{homeomorfna}. Oznaka $X \approx Y$.

$f: X \to Y$ je \textbf{odprta}, če je slika vsake odprte množice odprta. 

$f: X \to Y$ je \textbf{zaprta}, če je slika vsake zaprte množice zaprta. 

Naslednje trditve so ekvivalentne:
\begin{itemize}
	\item $f: X \to Y$ je homeomorfizem
	\item $f$ je zvezna bijekcija in $f^{-1}$ je zvezna
	\item $f$ je zvezna in odprta bijekcija
	\item $f$ je zvezna in zaprta bijekcija
\end{itemize}

\section*{Kompaktnost}
\textbf{Odprto pokritje} množice $X$ je vsaka družina (odprtih množic) $\mathcal{U} \subseteq \tau$, katere unija je cel $X$.

Prostor $X$ je \textbf{kompakten}, če v vsakem odprtem pokritju $X$ obstaja končno podpokritje.

\begin{itemize}
	\item Vsaka končna množica je kompaktna.
	\item V metričnem prostoru je vsaka kompaktna množica omejena.
\end{itemize}

\[ A^{\text{zap}} \subseteq X^{\text{kompakten}} \implies A \text{ kompakten} \]

\emph{Heine-Borel-Lebesgue:}
\[ A \subseteq \mathbb{R}^n \text{ je kompakten } \iff A \text{ zaprt in omejen} \]

V kompaktnem prostoru ima vsaka neskončna množica vsaj eno stekališče.

\emph{Bolzano-Weierstrass:}\\
Vsako omejeno zaporedje v $\mathbb{R}^n$ ima konvergentno podzaporedje.

Zvezna slika kompakta je kompakt.\\
$f: X \to Y$ zvezna, $A^\text{kompkt} \subseteq X$ $\implies$ $f(A)$ kompakt

$X$ kompakten $\iff$ v vsaki družini zap. podmnožic $X$, ki ima prazen presek, obstaja končna podmnožica, ki ima prazen presek.

\section*{Povezanost}
\textbf{Separacija} množice $X$ je razdelitev $X = A \amalg B$ na deve disjunktni, neprazni, odprti podmnožici.

Prostor, ki ima separacijo je \textbf{nepovezan}, sicer pa je \textbf{povezan}.

Alternativna definicija:
\begin{itemize}
	\item $X$ je povezan, če ga ni mogoče razdeliti na dve disjunktni neprazni množici
	\item $X$ je povezan, če sta njegovi edini podmnožici, ki sta zaprti in odprti hkrati, $\emptyset$ in $X$.
\end{itemize}

Povezane množice v $\mathbb{R}$ so natanko intervali.

Zvezna funkcije ohranjajo povezanost.\\
$f: X \to Y$ zvezna, $X$ povezana $\implies$ $f(X)$ povezana

$X$ je \textbf{povezan s potmi}, če za polubna $a,b \in X$ obstaja \textbf{pot} $p: [0,1] \to X$, zvezna, $p(0) = a$, $p(1) = b$.

$X$ povezan s potmi $\implies$ $X$ povezan

Če je $L$ povezan in je $L \subseteq M \subseteq \bar{L}$, je tudi $M$ povezan.